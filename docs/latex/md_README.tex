Ferramenta de desenho gráfico em C. Com suporte apenas para os arquivos P\+PM no formato P3.

\subsection*{Documentação}

O projeto possui uma documentação feita com o auxílio do \href{http://www.doxygen.nl/}{\tt Doxygen}, e pode ser encontrada \mbox{[}aqui\mbox{]}(docs). Basta executar localmente o arquivo {\ttfamily index.\+html} que se encontra na pasta {\ttfamily /html}. É possível enxergar detalhes sobre a implementação de cada parte do código.

\subsection*{Tipos de dados utilizados}


\begin{DoxyItemize}
\item typedef
\item struct
\item int
\item char
\item void
\end{DoxyItemize}

\subsection*{Considerações iniciais}

O programa possui um arquivo chamado {\ttfamily primitives}, em que nele são especificadas as primitivas suportadas descritas mais adiante.

O arquivo deve possuir uma formatação como específicado neste exemplo\+: 
\begin{DoxyCode}
image 600 400
clear 0 0 0
color 100 170 200
line 0 400 600 200
polygon 3 0 400 300 200 600 400
circle 200 100 50
color 180 30 50
fill 300 300
color 255 0 0
fill 0 0
flip vertical
flip horizontal
move right 10
move left 10
move top 10
move bottom 10
save test.ppm
\end{DoxyCode}


\subsection*{Como compilar o programa}


\begin{DoxyCode}
gcc ./modules/auxiliary.c ./modules/primitives.c ./modules/verification.c main.c -o main
\end{DoxyCode}


\subsection*{Como executar o programa}


\begin{DoxyCode}
./main
\end{DoxyCode}


\subsection*{Primitivas suportadas}

\#\#\# Criar uma nova “imagem”, com a largura e altura especificadas 
\begin{DoxyCode}
image 100 100
\end{DoxyCode}
 \begin{quote}
Irá criar uma imagem com o nome padrão \char`\"{}image.\+ppm\char`\"{} nas dimensões especificadas, largura e altura, respectivamente \end{quote}


\#\#\# Salvar a imagem atual em um arquivo usando o formato ppm 
\begin{DoxyCode}
save imageTest.ppm
\end{DoxyCode}
 \begin{quote}
Irá salvar uma imagem com o nome especificado \end{quote}


\#\#\# Abre uma imagem no formato ppm e carrega essa imagem no programa para futuras operações de desenho 
\begin{DoxyCode}
open imageTest.ppm
\end{DoxyCode}
 \begin{quote}
Irá abrir um arquivo P\+PM para operações \end{quote}


\#\#\# Especifica os dois pontos das extremidades da linha a ser desenhada, cada um com suas coordenadas (x, y) 
\begin{DoxyCode}
line 0 400 600 200
\end{DoxyCode}


\#\#\# Desenha um círculo nas posições x, y e tamanho especificados 
\begin{DoxyCode}
circle 200 100 50
\end{DoxyCode}


\#\#\# Desenha um polígono delimitado por uma lista de pontos 
\begin{DoxyCode}
polygon 3 0 400 300 200 600 400
\end{DoxyCode}


\#\#\# Limpa a imagem, setando todos os pixels para a cor especificada 
\begin{DoxyCode}
clear 0 0 0
\end{DoxyCode}


\#\#\# Muda a cor atual para uma cor especificada 
\begin{DoxyCode}
color 100 170 200
\end{DoxyCode}


\#\#\# Vira a imagem na vertical ou na horizontal 
\begin{DoxyCode}
flip vertical
flip horizontal
\end{DoxyCode}


\#\#\# Move a imagem em uma quantidade de pixels para direita, esquerda, cima ou baixo 
\begin{DoxyCode}
move right 10
move left 10
move top 10
move bottom 10
\end{DoxyCode}


\#\#\# Rotacionar a imagem em um angulo de 90 graus 
\begin{DoxyCode}
rotate 90
\end{DoxyCode}


\#\#\# Pintar o poligono ou o background 
\begin{DoxyCode}
fill 50 50
\end{DoxyCode}


\subsection*{Contribuidores ✨}

\tabulinesep=1mm
\begin{longtabu} spread 0pt [c]{*{2}{|X[-1]}|}
\hline
\PBS\centering \href{https://github.com/SamuelCost}{\tt \textsubscript{{\bfseries Samuel Costa (Samuel\+Cost)}} }  &\PBS\centering \href{https://github.com/durvaal}{\tt \textsubscript{{\bfseries Paulo Lima (Durvaal)}} }   \\\cline{1-2}
\end{longtabu}
